\documentclass[conference]{IEEEtran}
\IEEEoverridecommandlockouts
% The preceding line is only needed to identify funding in the first footnote. If that is unneeded, please comment it out.
\usepackage{cite}
\usepackage{amsmath,amssymb,amsfonts}
\usepackage{algorithmic}
\usepackage{graphicx}
\usepackage{textcomp}
\usepackage{xcolor}
\def\BibTeX{{\rm B\kern-.05em{\sc i\kern-.025em b}\kern-.08em
    T\kern-.1667em\lower.7ex\hbox{E}\kern-.125emX}}
\begin{document}

\title{Automatic Question Generation from Handwritten Lecture Notes on
KeyBERT-indexed T5-TrOCR Pipeline with Gemini Context Correction*\\
\author{\IEEEauthorblockN{Rommel John Ronduen\textsuperscript{1}, Jan Adrian Manzanero\textsuperscript{2}, Analyn Yumang\textsuperscript{3}}
\IEEEauthorblockA{\textit{School of Electrical, Electronics, and Computer Engineering} \\
\textit{Mapua University}\\
Manila, Philippines \\
\textsuperscript{1}rjhronduen@mymail.mapua.edu.ph\\ 
\textsuperscript{2}jacmanzanero@mymail.mapua.edu.ph\\ 
\textsuperscript{3}anyumang@mapua.edu.ph}}
}

\maketitle

\begin{abstract}
In this research, a system capable of performing optical character recognition
(OCR) on handwritten lecture notes and consequently using the extracted text for
automatic question genertion (AQG) was conceived. The system utilized the Text-to-Text
Transformer (T5) for AQG and TrOCR for OCR. The base version T5 was fine-tuned
using the SQuADv1.1 dataset while the pre-trained handwritten base version for
TrOCR was used. The system was evaluated using the word error rate (WER) for
OCR evaluation, while the Recall-Oriented Understudy for Gisting Evaluation (ROUGE)
and Bilingual Language Understanding Evaluation (BLEU) were used for AQG evaluation. 
The system achieved a WER of 0.40 and a question validity rate of 70\%.
\end{abstract}

\begin{IEEEkeywords}
Large Language Model, Optical Character Recognition,
Automatic Question Generation, Handwritten Lecture Notes, Raspberry Pi
\end{IEEEkeywords}

\section{Introduction}
Handwritten lecture notes are still widely used in educational institutions.
Since the advent of artifical intelligence (AI), education has become
the primary focus of AI research. AI has been used to automate several processes,
and one which is the generation of learning materials such as questions in the
form of quizzes. Literature defines automatic question generation (AQG) as the
process of generating questions from a given text. 

Despite existing implementations of AQG, no attempt has been given for utilizing 
handwritten lecture notes as a source for AQG. There have been several
approaches to AQG, such as rule-based, template-based, and neural-based,
but none of these approaches have been applied to handwritten lecture notes as 
a direct source for AQG through image capturing and OCR. It is noted by (Arbaaeen) AQG is defined by the methods used to generate questions.
The AQG system is composed of three main components: the context extraction and
parsing, the question generation, and the question validation. AQG systems differ in the method of extracting information and also 
the type of information that is being extracted. In fact, it was noted by them 
that it is suggested that AQG implementations utilize other forms of media 
sources for context bases. For instance, (Gaur) conceived 
of an AQG system that utilizes programming source code as the context for question
generation based on selected keywords from the code snippets. However, these 
code snippets were assumed to be transcribed as an input to the system and not 
from the handwritten medium. Another use case
through (Ou) involved the use of recorded videos as a source for AQG 
that led to question-answer pairs through the 
Bidirection and Auto-Regressive Transformer (BART) model for
the search of sentences in the text. They utilized an algorithm of indexing or 
searching through the extracted context for specific question generations, which 
will be explained further in the succeeding text. Despite this, the researchers utilized 
a recorded video as a source. On the other hand, (Moron) 
allowed for an implementation of AQG through the use of named entity recognition 
(NER) and semantic rules for question generation in aiding the learning of 
English that led to the generation of questions that primarily answered the
questions of "what" and "who". However, the AQG implementation relied on the 
manual input of the user in text boxes for the context. In total, AQG systems 
are capable of generating questions from a given context, but no system has been
developed to generate questions from handwritten lecture notes.
There exists a variety in the algorithms used for AQG, such as the use of 
the BART model for question generation through the use of a search algorithm and 
the use of NER and semantic rules for question generation. An example includes 
the use of treebanks for providing for inferences (Padilla). However, one such study
by (Tsai) utilized the use of the Text-to-Text Transformer (T5) model for AQG and
the evaluation utilizing the Recall-Oriented Understudy for Gisting Evaluation (ROUGE)
and the Bilingual Evaluation Understudy (BLEU) metrics. 
The Stanford Question Answering Dataset (SQuAD) was used for the 
fine-tuning of the T5 model that led to satisfactory similarity of model-generated
questions to the ground truth questions with a ROUGE-L score of 0.613. However,
they noted a limitation as the model had a BLEU score of 0.567 due to the processes
of the T5 model mixing the syntax and form of the context used for AQG. Due to 
the capability of the T5 model to generate questions from a given context while 
being fine-tuned to a specific dataset, it was chosen as the model for AQG in this 
research. In bridging the gap for AQG from handwritten lecture notes,
the Transformer-based Optical Character Recognition (TrOCR) model was used for the optical character recognition (OCR) of the 
handwritten lecture notes. Starting first with OCR, it is often associated with 
convolutional neural networks (CNN) and recurrent neural networks (RNN). As noted 
by (Manlises) the use of CNNs for OCR is often used for the detection of text
in images and even objects such as that of the implementation for the detection 
of the different types of mushrooms (Caya). CNNs have also been used for 
recognizing text through the transformation of shorthand terminologies to 
English text (Vitug). In fact, it was shown by (Ligsay) that it was 
possible to recognize text in Baybayin (a Filipino lettering system) 
using CNNs. Another involved the identification of expiry dates on 
canned goods (Manlises). In terms of evaluation, these CNNs are often evaluated 
using confusion matrices. These confusion matrices are used to evaluate the
performance of the OCR model in terms of the true positive, true negative, false
positive, and false negative values (Villaverde). However, a transformer-based
approach has been used for OCR, such as the TrOCR model (Li) which has been used in recognition of text from 
scanned receipts (Zhang) and even in the recognition of text from images of 
Arabic text (Mortadi). The model has been exemplary over the use of CNNs by its 
encoder-decoder framework with pre-trained weights for the recognition of text.
\\
\indent In lieu of the gap on AQG from utilizing handwritten lecture notes, 
the general objective of this research to develop a system of allowing for
automatic question generation from handwritten lecture notes on KeyBERT-indexed 
T5-TrOCR pipeline with Gemini context correction. The specific objectives of
this research to utilize a fine-tuned T5 model for AQG on SQuADv1.1 while 
using KeyBERT for indexing the context of the handwritten lecture notes; to 
utilize the base handwritten TrOCR model for the OCR of the handwritten lecture;
to evaluate the system using the word error rate (WER) for OCR evaluation and
the ROUGE and BLEU metrics for AQG evaluation; and to utilize a Raspberry Pi 5
within a constructed enclosure with proper illumination for the 
facilitation and procurement of the system processes, a web camera for image
capture, and a touchscreen monitor for the display of the generated questions.
\\ 
\indent This research is mainly limited by the consideration of only 
single-column, diagram and equation free, and English handwritten lecture notes 
of no erasures on strictly letter-sized plain sheet paper.
Moreover, the system is confined to the 
use of the T5-TrOCR pipeline utilizing the base versions while 
also utilizing the SQuADv1.1 dataset for the fine-tuning. It is also important 
to note that the system is limited to the facilitation of the processes in 
the hardware of the 8-gigabyte version of the Raspberry Pi 5. The use of 
a higher parameter count for the T5 and TrOCR models and the use of a different 
dataset for fine-tuning may allow for differing results. Moreover, the 
utilization of a different hardware setup may also lead to alternative, if not, 
better results in terms of processing time. 



\section{Materials and Methods}
    \subsection{Hardware Development}
        \subsubsection{System Block Diagram}
        \subsubsection{Experimental Setup}
    \subsection{Software Development}
        \subsubsection{System Flowchart}
        \subsubsection{Model Fine-tuning}
    \subsection{Data Gathering}
    \subsection{Testing and Evaluation}

\section{Results and Discussion}
\section{Conclusion and Recommendations}

\subsection{Maintaining the Integrity of the Specifications}

The IEEEtran class file is used to format your paper and style the text. All margins, 
column widths, line spaces, and text fonts are prescribed; please do not 
alter them. You may note peculiarities. For example, the head margin
measures proportionately more than is customary. This measurement 
and others are deliberate, using specifications that anticipate your paper 
as one part of the entire proceedings, and not as an independent document. 
Please do not revise any of the current designations.

\section{Prepare Your Paper Before Styling}
Before you begin to format your paper, first write and save the content as a 
separate text file. Complete all content and organizational editing before 
formatting. Please note sections \ref{AA}--\ref{SCM} below for more information on 
proofreading, spelling and grammar.

Keep your text and graphic files separate until after the text has been 
formatted and styled. Do not number text heads---{\LaTeX} will do that 
for you.

\subsection{Abbreviations and Acronyms}\label{AA}
Define abbreviations and acronyms the first time they are used in the text, 
even after they have been defined in the abstract. Abbreviations such as 
IEEE, SI, MKS, CGS, ac, dc, and rms do not have to be defined. Do not use 
abbreviations in the title or heads unless they are unavoidable.

\subsection{Units}
\begin{itemize}
\item Use either SI (MKS) or CGS as primary units. (SI units are encouraged.) English units may be used as secondary units (in parentheses). An exception would be the use of English units as identifiers in trade, such as ``3.5-inch disk drive''.
\item Avoid combining SI and CGS units, such as current in amperes and magnetic field in oersteds. This often leads to confusion because equations do not balance dimensionally. If you must use mixed units, clearly state the units for each quantity that you use in an equation.
\item Do not mix complete spellings and abbreviations of units: ``Wb/m\textsuperscript{2}'' or ``webers per square meter'', not ``webers/m\textsuperscript{2}''. Spell out units when they appear in text: ``. . . a few henries'', not ``. . . a few H''.
\item Use a zero before decimal points: ``0.25'', not ``.25''. Use ``cm\textsuperscript{3}'', not ``cc''.)
\end{itemize}

\subsection{Equations}
Number equations consecutively. To make your 
equations more compact, you may use the solidus (~/~), the exp function, or 
appropriate exponents. Italicize Roman symbols for quantities and variables, 
but not Greek symbols. Use a long dash rather than a hyphen for a minus 
sign. Punctuate equations with commas or periods when they are part of a 
sentence, as in:
\begin{equation}
a+b=\gamma\label{eq}
\end{equation}

Be sure that the 
symbols in your equation have been defined before or immediately following 
the equation. Use ``\eqref{eq}'', not ``Eq.~\eqref{eq}'' or ``equation \eqref{eq}'', except at 
the beginning of a sentence: ``Equation \eqref{eq} is . . .''

\subsection{\LaTeX-Specific Advice}

Please use ``soft'' (e.g., \verb|\eqref{Eq}|) cross references instead
of ``hard'' references (e.g., \verb|(1)|). That will make it possible
to combine sections, add equations, or change the order of figures or
citations without having to go through the file line by line.

Please don't use the \verb|{eqnarray}| equation environment. Use
\verb|{align}| or \verb|{IEEEeqnarray}| instead. The \verb|{eqnarray}|
environment leaves unsightly spaces around relation symbols.

Please note that the \verb|{subequations}| environment in {\LaTeX}
will increment the main equation counter even when there are no
equation numbers displayed. If you forget that, you might write an
article in which the equation numbers skip from (17) to (20), causing
the copy editors to wonder if you've discovered a new method of
counting.

{\BibTeX} does not work by magic. It doesn't get the bibliographic
data from thin air but from .bib files. If you use {\BibTeX} to produce a
bibliography you must send the .bib files. 

{\LaTeX} can't read your mind. If you assign the same label to a
subsubsection and a table, you might find that Table I has been cross
referenced as Table IV-B3. 

{\LaTeX} does not have precognitive abilities. If you put a
\verb|\label| command before the command that updates the counter it's
supposed to be using, the label will pick up the last counter to be
cross referenced instead. In particular, a \verb|\label| command
should not go before the caption of a figure or a table.

Do not use \verb|\nonumber| inside the \verb|{array}| environment. It
will not stop equation numbers inside \verb|{array}| (there won't be
any anyway) and it might stop a wanted equation number in the
surrounding equation.

\subsection{Some Common Mistakes}\label{SCM}
\begin{itemize}
\item The word ``data'' is plural, not singular.
\item The subscript for the permeability of vacuum $\mu_{0}$, and other common scientific constants, is zero with subscript formatting, not a lowercase letter ``o''.
\item In American English, commas, semicolons, periods, question and exclamation marks are located within quotation marks only when a complete thought or name is cited, such as a title or full quotation. When quotation marks are used, instead of a bold or italic typeface, to highlight a word or phrase, punctuation should appear outside of the quotation marks. A parenthetical phrase or statement at the end of a sentence is punctuated outside of the closing parenthesis (like this). (A parenthetical sentence is punctuated within the parentheses.)
\item A graph within a graph is an ``inset'', not an ``insert''. The word alternatively is preferred to the word ``alternately'' (unless you really mean something that alternates).
\item Do not use the word ``essentially'' to mean ``approximately'' or ``effectively''.
\item In your paper title, if the words ``that uses'' can accurately replace the word ``using'', capitalize the ``u''; if not, keep using lower-cased.
\item Be aware of the different meanings of the homophones ``affect'' and ``effect'', ``complement'' and ``compliment'', ``discreet'' and ``discrete'', ``principal'' and ``principle''.
\item Do not confuse ``imply'' and ``infer''.
\item The prefix ``non'' is not a word; it should be joined to the word it modifies, usually without a hyphen.
\item There is no period after the ``et'' in the Latin abbreviation ``et al.''.
\item The abbreviation ``i.e.'' means ``that is'', and the abbreviation ``e.g.'' means ``for example''.
\end{itemize}
An excellent style manual for science writers is \cite{b7}.

\subsection{Authors and Affiliations}
\textbf{The class file is designed for, but not limited to, six authors.} A 
minimum of one author is required for all conference articles. Author names 
should be listed starting from left to right and then moving down to the 
next line. This is the author sequence that will be used in future citations 
and by indexing services. Names should not be listed in columns nor group by 
affiliation. Please keep your affiliations as succinct as possible (for 
example, do not differentiate among departments of the same organization).

\subsection{Identify the Headings}
Headings, or heads, are organizational devices that guide the reader through 
your paper. There are two types: component heads and text heads.

Component heads identify the different components of your paper and are not 
topically subordinate to each other. Examples include Acknowledgments and 
References and, for these, the correct style to use is ``Heading 5''. Use 
``figure caption'' for your Figure captions, and ``table head'' for your 
table title. Run-in heads, such as ``Abstract'', will require you to apply a 
style (in this case, italic) in addition to the style provided by the drop 
down menu to differentiate the head from the text.

Text heads organize the topics on a relational, hierarchical basis. For 
example, the paper title is the primary text head because all subsequent 
material relates and elaborates on this one topic. If there are two or more 
sub-topics, the next level head (uppercase Roman numerals) should be used 
and, conversely, if there are not at least two sub-topics, then no subheads 
should be introduced.

\subsection{Figures and Tables}
\paragraph{Positioning Figures and Tables} Place figures and tables at the top and 
bottom of columns. Avoid placing them in the middle of columns. Large 
figures and tables may span across both columns. Figure captions should be 
below the figures; table heads should appear above the tables. Insert 
figures and tables after they are cited in the text. Use the abbreviation 
``Fig.~\ref{fig}'', even at the beginning of a sentence.

\begin{table}[htbp]
\caption{Table Type Styles}
\begin{center}
\begin{tabular}{|c|c|c|c|}
\hline
\textbf{Table}&\multicolumn{3}{|c|}{\textbf{Table Column Head}} \\
\cline{2-4} 
\textbf{Head} & \textbf{\textit{Table column subhead}}& \textbf{\textit{Subhead}}& \textbf{\textit{Subhead}} \\
\hline
copy& More table copy$^{\mathrm{a}}$& &  \\
\hline
\multicolumn{4}{l}{$^{\mathrm{a}}$Sample of a Table footnote.}
\end{tabular}
\label{tab1}
\end{center}
\end{table}

\begin{figure}[htbp]
\centerline{\includegraphics{fig1.png}}
\caption{Example of a figure caption.}
\label{fig}
\end{figure}

Figure Labels: Use 8 point Times New Roman for Figure labels. Use words 
rather than symbols or abbreviations when writing Figure axis labels to 
avoid confusing the reader. As an example, write the quantity 
``Magnetization'', or ``Magnetization, M'', not just ``M''. If including 
units in the label, present them within parentheses. Do not label axes only 
with units. In the example, write ``Magnetization (A/m)'' or ``Magnetization 
\{A[m(1)]\}'', not just ``A/m''. Do not label axes with a ratio of 
quantities and units. For example, write ``Temperature (K)'', not 
``Temperature/K''.

\section*{Acknowledgment}

The preferred spelling of the word ``acknowledgment'' in America is without 
an ``e'' after the ``g''. Avoid the stilted expression ``one of us (R. B. 
G.) thanks $\ldots$''. Instead, try ``R. B. G. thanks$\ldots$''. Put sponsor 
acknowledgments in the unnumbered footnote on the first page.

\section*{References}

Please number citations consecutively within brackets \cite{b1}. The 
sentence punctuation follows the bracket \cite{b2}. Refer simply to the reference 
number, as in \cite{b3}---do not use ``Ref. \cite{b3}'' or ``reference \cite{b3}'' except at 
the beginning of a sentence: ``Reference \cite{b3} was the first $\ldots$''

Number footnotes separately in superscripts. Place the actual footnote at 
the bottom of the column in which it was cited. Do not put footnotes in the 
abstract or reference list. Use letters for table footnotes.

Unless there are six authors or more give all authors' names; do not use 
``et al.''. Papers that have not been published, even if they have been 
submitted for publication, should be cited as ``unpublished'' \cite{b4}. Papers 
that have been accepted for publication should be cited as ``in press'' \cite{b5}. 
Capitalize only the first word in a paper title, except for proper nouns and 
element symbols.

For papers published in translation journals, please give the English 
citation first, followed by the original foreign-language citation \cite{b6}.

\begin{thebibliography}{00}
\bibitem{b1} G. Eason, B. Noble, and I. N. Sneddon, ``On certain integrals of Lipschitz-Hankel type involving products of Bessel functions,'' Phil. Trans. Roy. Soc. London, vol. A247, pp. 529--551, April 1955.
\bibitem{b2} J. Clerk Maxwell, A Treatise on Electricity and Magnetism, 3rd ed., vol. 2. Oxford: Clarendon, 1892, pp.68--73.
\bibitem{b3} I. S. Jacobs and C. P. Bean, ``Fine particles, thin films and exchange anisotropy,'' in Magnetism, vol. III, G. T. Rado and H. Suhl, Eds. New York: Academic, 1963, pp. 271--350.
\bibitem{b4} K. Elissa, ``Title of paper if known,'' unpublished.
\bibitem{b5} R. Nicole, ``Title of paper with only first word capitalized,'' J. Name Stand. Abbrev., in press.
\bibitem{b6} Y. Yorozu, M. Hirano, K. Oka, and Y. Tagawa, ``Electron spectroscopy studies on magneto-optical media and plastic substrate interface,'' IEEE Transl. J. Magn. Japan, vol. 2, pp. 740--741, August 1987 [Digests 9th Annual Conf. Magnetics Japan, p. 301, 1982].
\bibitem{b7} M. Young, The Technical Writer's Handbook. Mill Valley, CA: University Science, 1989.
\end{thebibliography}
\vspace{12pt}
\color{red}
IEEE conference templates contain guidance text for composing and formatting conference papers. Please ensure that all template text is removed from your conference paper prior to submission to the conference. Failure to remove the template text from your paper may result in your paper not being published.

\end{document}
